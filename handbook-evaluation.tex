\hh{Run evaluation}

\emph{Run} is a solution to a problem submitted for judging.
The size of the source file with the run may not exceed 256KB.

Immediately after submission of any run, the team may continue working on
other problems.

Contest software evaluates each run and marks it as \emph{accepted} or \emph{rejected}.

The run is evaluated by executing it on a secret set of tests, common for
all contestants.
A run is accepted only if it gives correct answers to all tests.

The \emph{memory limit} is the maximum amount of memory that a run may
utilize on each test.
The \emph{time limit} is the maximum execution time per test.
The time and memory limits for each problem are specified in the problem statements.
The run is not accepted if the program exceeds these limits.

As soon as the run is evaluated, the contest software displays evaluation results.
The team is informed whether the run is accepted or not.
If the run is rejected, the error type and the test number (when applicable)
are indicated.

All tests cases are numbered from one. The first test cases in the test set
are equal to the sample tests from the problem statement.
The following tests are ordered with the idea to make easier test cases
come before harder ones, although there are no guarantees.

The possible outcomes are listed in the table at the next page in their order of priority.
For example, if runtime error has occurred, then output is not checked.

Evaluation process may be stopped several minutes before the end of the
Contest.
All runs submitted after this moment will be evaluated just after the end
of the Contest.

Runs are evaluated on \emph{Intel Core i3-8100, 3.6GHz} computers under
\emph{Windows 10}.

Runs are not allowed to:
\begin{itemize}
    \item access the network;
    \item perform any file or network I/O;
    \item execute other programs and create new processes;
    \item work with subdirectories;
    \item create or manipulate any GUI resources (windows, dialog boxes, etc.);
    \item work with external devices (sound, printer, etc.);
    \item attack system security;
    \item do anything else that can stir the evaluation process and the Contest.
\end{itemize}


\begin{tabular}{|p{3.2cm}|p{1.45cm}|p{4.5cm}|p{6.5cm}|} \hline
\raggedright \header{Outcome} & \raggedright \header{Test Number} & \raggedright \header{Comment} & {\raggedright \header{Possible Reasons}} \\ \hline\raggedright
        Compilation error
            & \raggedright No
            & \raggedright Executable file was not created after compilation.
            & {\raggedright
                \qitem Syntax error in the program;
                \qitem wrong file extension or language specified.
            } \\ \hline\raggedright
        Security violation
            & \raggedright Yes
            & \raggedright The program tried to violate the Contest Rules.
            & {\raggedright
                \qitem Error in the program;
                \qitem purposeful rules violation (the violating team is disqualified in this case).
            } \\ \hline\raggedright
        Runtime error
            & \raggedright Yes
            & \raggedright The program terminates with non-zero exit code or throws an uncaught OS exception.
            & {\raggedright
                \qitem Runtime error;
                \qitem uncaught exception;
                \qitem missing '\verb|return 0|' statement in C++ main function;
                \qitem '\verb|return (non-zero)|' statement in C++ main function;
                \qitem '\verb|System.exit(non-zero)|' in Java.
            } \\ \hline\raggedright
        Time limit exceeded
            & \raggedright Yes
            & \raggedright The program exceeds the time limit.
            & {\raggedright
                \qitem Inefficient solution;
                \qitem error in the program.
            } \\ \hline\raggedright
        Memory limit exceeded
            & \raggedright Yes
            & \raggedright The program exceeds the memory limit.
            & {\raggedright
                \qitem Inefficient solution;
                \qitem error in the program.
            } \\ \hline\raggedright
        Idleness limit exceeded
            & \raggedright Yes
            & \raggedright The program does not consume processor time for a long period.
            & {\raggedright
                \qitem Input from console;
                \qitem error in the program.
            } \\ \hline\raggedright
        Wrong answer
            & \raggedright Yes
            & \raggedright The answer is not correct or the checker cannot check output because it does not match the format specified in the problem statement.
            & {\raggedright
                \qitem{}The algorithm is not correct;
                \qitem{}output format is not correct;
                \qitem{}no output file.
            } \\ \hline\raggedright
        Accepted
            & \raggedright No
            & \raggedright Run is accepted.
            & {\raggedright Program is correct.} \\ \hline
\end{tabular}
